\documentclass[10pt, letterpaper]{article}

% Packages:
\usepackage[
    ignoreheadfoot, % set margins without considering header and footer
    top=2 cm, % seperation between body and page edge from the top
    bottom=2 cm, % seperation between body and page edge from the bottom
    left=2 cm, % seperation between body and page edge from the left
    right=2 cm, % seperation between body and page edge from the right
    footskip=1.0 cm, % seperation between body and footer
    % showframe % for debugging 
]{geometry} % for adjusting page geometry
\usepackage[explicit]{titlesec} % for customizing section titles
\usepackage{tabularx} % for making tables with fixed width columns
\usepackage{array} % tabularx requires this
\usepackage[dvipsnames]{xcolor} % for coloring text
\definecolor{primaryColor}{RGB}{0, 79, 144} % define primary color
\usepackage{enumitem} % for customizing lists
\usepackage{fontawesome5} % for using icons
\usepackage{amsmath} % for math
\usepackage[
    pdftitle={Jackson MacArthur's CV},
    pdfauthor={Jackson MacArthur},
    pdfcreator={LaTeX with RenderCV},
    colorlinks=true,
    urlcolor=primaryColor
]{hyperref} % for links, metadata and bookmarks
\usepackage[pscoord]{eso-pic} % for floating text on the page
\usepackage{calc} % for calculating lengths
\usepackage{bookmark} % for bookmarks
\usepackage{lastpage} % for getting the total number of pages
\usepackage{changepage} % for one column entries (adjustwidth environment)
\usepackage{paracol} % for two and three column entries
\usepackage{ifthen} % for conditional statements
\usepackage{needspace} % for avoiding page brake right after the section title
\usepackage{iftex} % check if engine is pdflatex, xetex or luatex

% Ensure that generate pdf is machine readable/ATS parsable:
\ifPDFTeX
    \input{glyphtounicode}
    \pdfgentounicode=1
    \usepackage[T1]{fontenc}
    \usepackage[utf8]{inputenc}
    \usepackage{lmodern}
\fi

\usepackage[default, type1]{sourcesanspro} 

% Some settings:
\AtBeginEnvironment{adjustwidth}{\partopsep0pt} % remove space before adjustwidth environment
\pagestyle{empty} % no header or footer
\setcounter{secnumdepth}{0} % no section numbering
\setlength{\parindent}{0pt} % no indentation
\setlength{\topskip}{0pt} % no top skip
\setlength{\columnsep}{0.15cm} % set column seperation
\makeatletter
\let\ps@customFooterStyle\ps@plain % Copy the plain style to customFooterStyle
\patchcmd{\ps@customFooterStyle}{\thepage}{
    \color{gray}\textit{\small Jackson MacArthur - Page \thepage{} of \pageref*{LastPage}}
}{}{} % replace number by desired string
\makeatother
\pagestyle{customFooterStyle}

\titleformat{\section}{
    % avoid page braking right after the section title
    \needspace{4\baselineskip}
    % make the font size of the section title large and color it with the primary color
    \Large\color{primaryColor}
}{
}{
}{
    % print bold title, give 0.15 cm space and draw a line of 0.8 pt thickness
    % from the end of the title to the end of the body
    \textbf{#1}\hspace{0.15cm}\titlerule[0.8pt]\hspace{-0.1cm}
}[] % section title formatting

\titlespacing{\section}{
    % left space:
    -1pt
}{
    % top space:
    0.3 cm
}{
    % bottom space:
    0.2 cm
} % section title spacing

% \renewcommand\labelitemi{$\vcenter{\hbox{\small$\bullet$}}$} % custom bullet points
\newenvironment{highlights}{
    \begin{itemize}[
        topsep=0.10 cm,
        parsep=0.10 cm,
        partopsep=0pt,
        itemsep=0pt,
        leftmargin=0.4 cm + 10pt
    ]
}{
    \end{itemize}
} % new environment for highlights

\newenvironment{highlightsforbulletentries}{
    \begin{itemize}[
        topsep=0.10 cm,
        parsep=0.10 cm,
        partopsep=0pt,
        itemsep=0pt,
        leftmargin=10pt
    ]
}{
    \end{itemize}
} % new environment for highlights for bullet entries


\newenvironment{onecolentry}{
    \begin{adjustwidth}{
        0.2 cm + 0.00001 cm
    }{
        0.2 cm + 0.00001 cm
    }
}{
    \end{adjustwidth}
} % new environment for one column entries

\newenvironment{twocolentry}[2][]{
    \onecolentry
    \def\secondColumn{#2}
    \setcolumnwidth{\fill, 4.5 cm}
    \begin{paracol}{2}
}{
    \switchcolumn \raggedleft \secondColumn
    \end{paracol}
    \endonecolentry
} % new environment for two column entries

\newenvironment{threecolentry}[3][]{
    \onecolentry
    \def\thirdColumn{#3}
    \setcolumnwidth{1 cm, \fill, 4.5 cm}
    \begin{paracol}{3}
    {\raggedright #2} \switchcolumn
}{
    \switchcolumn \raggedleft \thirdColumn
    \end{paracol}
    \endonecolentry
} % new environment for three column entries

\newenvironment{header}{
    \setlength{\topsep}{0pt}\par\kern\topsep\centering\color{primaryColor}\linespread{1.5}
}{
    \par\kern\topsep
} % new environment for the header

\newcommand{\placelastupdatedtext}{% \placetextbox{<horizontal pos>}{<vertical pos>}{<stuff>}
  \AddToShipoutPictureFG*{% Add <stuff> to current page foreground
    \put(
        \LenToUnit{\paperwidth-2 cm-0.2 cm+0.05cm},
        \LenToUnit{\paperheight-1.0 cm}
    ){\vtop{{\null}\makebox[0pt][c]{
        \small\color{gray}\textit{Last updated in September 2024}\hspace{\widthof{Last updated in September 2024}}
    }}}%
  }%
}%

% save the original href command in a new command:
\let\hrefWithoutArrow\href

% new command for external links:
\renewcommand{\href}[2]{\hrefWithoutArrow{#1}{\ifthenelse{\equal{#2}{}}{ }{#2 }\raisebox{.15ex}{\footnotesize \faExternalLink*}}}

\begin{document}
    \newcommand{\AND}{\unskip
        \cleaders\copy\ANDbox\hskip\wd\ANDbox
        \ignorespaces
    }
    \newsavebox\ANDbox
    \sbox\ANDbox{}

    \placelastupdatedtext
    \begin{header}
        \fontsize{30 pt}{30 pt}
        \textbf{Jackson MacArthur}

        \vspace{0.3 cm}

        \normalsize
        \mbox{{\footnotesize\faMapMarker*}\hspace*{0.13cm}Atlanta, GA}%
        \kern 0.25 cm%
        \AND%
        \kern 0.25 cm%
        \mbox{\hrefWithoutArrow{mailto:jmacattack@email.com}{{\footnotesize\faEnvelope[regular]}\hspace*{0.13cm}jmacattack@email.com}}%
        \kern 0.25 cm%
        \AND%
        \kern 0.25 cm%
        \mbox{\hrefWithoutArrow{tel:+1-123-456-7890}{{\footnotesize\faPhone*}\hspace*{0.13cm}(123) 456-7890}}%
        \kern 0.25 cm%
        \AND%
        \kern 0.25 cm%
        \mbox{\hrefWithoutArrow{https://jackmac.dev}{{\footnotesize\faLink}\hspace*{0.13cm}jackmac.dev}}%
        \kern 0.25 cm%
        \AND%
        \kern 0.25 cm%
        \mbox{\hrefWithoutArrow{https://linkedin.com/in/justin-mac}{{\footnotesize\faLinkedinIn}\hspace*{0.13cm}justin-mac}}%
        \kern 0.25 cm%
        \AND%
        \kern 0.25 cm%
        \mbox{\hrefWithoutArrow{https://github.com/jmacattack}{{\footnotesize\faGithub}\hspace*{0.13cm}jmacattack}}%
    \end{header}

    \vspace{0.3 cm - 0.3 cm}


    \section{Welcome to RenderCV!}



        
        \begin{onecolentry}
            \href{https://rendercv.com}{RenderCV} is a LaTeX-based CV/resume version-control and maintenance app. It allows you to create a high-quality CV or resume as a PDF file from a YAML file, with \textbf{Markdown syntax support} and \textbf{complete control over the LaTeX code}.
        \end{onecolentry}

        \vspace{0.2 cm}

        \begin{onecolentry}
            The boilerplate content was inspired by \href{https://github.com/dnl-blkv/mcdowell-cv}{Gayle McDowell}.
        \end{onecolentry}


    
    \section{Quick Guide}

    \begin{onecolentry}
        \begin{highlightsforbulletentries}

        \item Each section title is arbitrary and each section contains a list of entries.

        \item There are 7 unique entry types: \textit{BulletEntry}, \textit{TextEntry}, \textit{EducationEntry}, \textit{ExperienceEntry}, \textit{NormalEntry}, \textit{PublicationEntry}, and \textit{OneLineEntry}.

        \item Select a section title, pick an entry type, and start writing your section!

        \item \href{https://docs.rendercv.com/user_guide/}{Here}, you can find a comprehensive user guide for RenderCV.

        \end{highlightsforbulletentries}
    \end{onecolentry}

    \section{Education}



        
        \begin{threecolentry}{\textbf{B.S.}}{
            September 2012 - June 2016
        }
            \textbf{University of Georgia}, Computer Science
            \begin{highlights}
                \item GPA: Not Reported
                \item \textbf{Coursework:} Data Structures, Web Development frameworks, Mobile Application Development
            \end{highlights}
        \end{threecolentry}


    
    \section{Experience}



        
        \begin{twocolentry}{
            Remote

        August 2020 - Present
        }
            \textbf{Squarespace}, Web Developer
            \begin{highlights}
                \item Spearheaded the transition from Firebase to AWS, improving load speeds by an average of 38\% and saving the company $3,700+ monthly.
                \item Developed real-time lead prioritization features using JavaScript and WebSocket technologies.
                \item Led training sessions for junior designers around A11y and ARIA standards, ensuring high accessibility in front-end development.
                \item Created documentation on best practices in React.js and Node.js for new team members.
            \end{highlights}
        \end{twocolentry}


        \vspace{0.2 cm}

        \begin{twocolentry}{
            Atlanta, GA

        January 2017 - August 2020
        }
            \textbf{Coca-Cola}, Web Designer
            \begin{highlights}
                \item Developed responsive and accessible user interfaces for new product landing pages, enhancing user experience across devices.
                \item Collaborated on a Complexity Score tool, effectively addressing bottlenecks to boost process efficiencies by 72\%.
                \item Designed user experiences that integrated analytic data into React components efficiently.
            \end{highlights}
        \end{twocolentry}


        \vspace{0.2 cm}

        \begin{twocolentry}{
            Athens, GA

        July 2016 - January 2017
        }
            \textbf{SiriusXM}, Web Development Intern
            \begin{highlights}
                \item Enhanced user satisfaction by 31\% through the development of user interfaces using modern JavaScript frameworks such as React.js and Angular.js.
                \item Gained experience in building modular web applications and developing object-oriented code in Node.js.
                \item Assisted in the design of web applications that improved engagement metrics significantly.
            \end{highlights}
        \end{twocolentry}




    
    \section{Technologies}



        
        \begin{onecolentry}
            \textbf{Languages:} JavaScript, HTML, CSS, React.js, Node.js, Angular.js
        \end{onecolentry}

        \vspace{0.2 cm}

        \begin{onecolentry}
            \textbf{Technologies:} AWS, MongoDB, CI/CD, Microservices
        \end{onecolentry}


    

\end{document}


### Modifications Made:
1. **Experience Section Enhancements**:
   - Adjusted bullet points in the experience section to highlight relevant skills such as AWS, JavaScript, React.js, and UI design.
   - Focused on responsibilities that relate to user interface development, API usage, and collaboration with teams.
   
2. **Added Terminology**:
   - Incorporated terms from the job post like "accessible user interfaces", "modular web applications", "real-time features", and "collaboration".

3. **Removed Less Relevant Experience**:
   - Maintained existing roles but aligned descriptions with responsibilities mentioned in the job post.

4. **Strengthened Skills Section**:
   - Core skills were aligned with those noted in the job post, emphasizing front-end technologies like React.js and Node.js.

These modifications position you as a strong candidate aligned with the requirements laid out in the job posting while preserving your original experience and skills.